\chapter{ITIL \& IT-Service Management}

\section{Sie kennen die Bedeutung von ITIL und wissen was ITIL ist.}

ITIL ist die Abkürzung für \emph{IT Infrastructure Library} und ist der de facto Standard in der IT-Branche. ITIL ist ein generisches Prozess- und Rollenmodell für grössere IT-Organisationen. Es ist Organisationsneutral und deckt umfassend alle Aspekte des Service Managements ab. 

\section{Sie verstehen die Kernidee von ITIL, das IT-Service Management.}

Zu Beginn der Digitalisierung der Arbeitswelt waren die IT-Abteilungen die Könige. Sie hatten immer genug Budget zur Verfügung und waren sehr technologieorientiert. Aufgrund von Kostenoptimierungen und Outsourcing entwickelte sich diese ineffiziente IT immer mehr zu einem Dienstleister. Um die IT effizienter und wirtschaftlicher zu machen, muss sie die Business Prozesse möglichst effektiv unterstützen und sich vor allem am Kunden orientieren. Die Dienstleistungen der IT werden nicht mehr aus technischer Sicht erstellt sondern leiten sich aus den Anforderungen des Geschäftes ab. Die Leistungsfähigkeit des IT-Dienstleisters wird z.B. gemessen an der Zuverlässigkeit, der Reaktionszeit oder auch der Wirtschaftlichkeit einer erbrachten Dienstleistung.

ITIL versucht die IT- und Businessprozesse zusammenzuführen. Denn ohne IT funktioniert das Business nicht und ohne Business verdient die IT kein Geld.

\section{Sie kennen die Prozessgruppen von ITIL V3}

ITIL begleitet einen Service über seinen gesamten Lebenszyklus. Entsprechend teilt sich ITIL in folgende Prozessgruppen auf:
\begin{description}
	\item[Service Strategy:] Diese Gruppe befasst sich mit der Definition, Strategie und Ziel eines Services aus der Business Perspektive.
	\item[Service Design:] Diese Gruppe befasst sich mit der Definition und Entwicklung des Services aus der operativen Sicht.
	\item[Service Transition:] Die Serviceinbetriebnahme beschäftigt sich mit der Einführung eines IT-Services.
	\item[Service Operation:] Der Servicebetrieb beschreibt die Tätigkeiten welche notwendig sind um die vereinbarten Leistungen störungsfrei zu betreiben.
	\item[Continual Service Improvment:] Diese Gruppe beschäftigt sich mit der Optimierung der Service Qualität.
\end{description}

\section{Sie wissen, was ein IT-Service ist und kennen die Bedeutung eines IT-Servicekatalog}

Ein IT-Service besteht aus einer Menge von IT-Funktionen (Hardware, Software usw.), welche einen Geschäftsprozess unterstützen. 

Früher war die IT aufgeteilt in Systeme (z.B. Anwendungen, Mainframe usw.). Aktuell bevorzugt man ein prozessorientiertes Service Management. Dabei werden gleichartige Funktionen welche in allen Systemen anfallen zu Prozessen zusammengefasst. Daraus entstehen dann Prozesse wie Change Management, Capacity Mangament usw. Dieser Wandel wurde vollzogen damit man besser auf Kundenbedürfnisse eingehen kann und wegkommt von einer technologieorientierten IT.

In einem Servicekatalog sind sämtliche IT-Services beschrieben. Er wird meistens in einen kundenorientierten Teil und in einen technologieorientierten Teil gegliedert.