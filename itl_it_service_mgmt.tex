\chapter{ITIL \& IT-Service Management}

\section{Sie kennen die Bedeutung von ITIL und wissen was ITIL ist.}

ITIL ist die Abkürzung für \emph{IT Infrastructure Library} und ist der de facto Standard in der IT-Branche. ITIL ist ein generisches Prozess- und Rollenmodell für grössere IT-Organisationen. Es ist Organisationsneutral und deckt umfassend alle Aspekte des Service Managements ab. 

\section{Sie verstehen die Kernidee von ITIL, das IT-Service Management.}

Zu Beginn der Digitalisierung der Arbeitswelt waren die IT-Abteilungen die Könige. Sie hatten immer genug Budget zur Verfügung und waren sehr technologieorientiert. Aufgrund von Kostenoptimierungen und Outsourcing entwickelte sich diese ineffiziente IT immer mehr zu einem Dienstleister. Um die IT effizienter und wirtschaftlicher zu machen, muss sie die Business Prozesse möglichst effektiv unterstützen und sich vor allem am Kunden orientieren. Die Dienstleistungen der IT werden nicht mehr aus technischer Sicht erstellt sondern leiten sich aus den Anforderungen des Geschäftes ab. Die Leistungsfähigkeit des IT-Dienstleisters wird z.B. gemessen an der Zuverlässigkeit, der Reaktionszeit oder auch der Wirtschaftlichkeit einer erbrachten Dienstleistung.

ITIL versucht die IT- und Businessprozesse zusammenzuführen. Denn ohne IT funktioniert das Business nicht und ohne Business verdient die IT kein Geld.

\section{Sie kennen die Prozessgruppen von ITIL V3}

ITIL begleitet einen Service über seinen gesamten Lebenszyklus. Entsprechend teilt sich ITIL in folgende Prozessgruppen auf:
\begin{description}
	\item[Service Strategy:] Diese Gruppe befasst sich mit der Definition, Strategie und Ziel eines Services aus der Business Perspektive. Die Prozesse richten sich an die Geschäftsleitung und das obere Management. Das Unternehmen muss zuerst die Fähigkeit zum Service Management entwickeln. Es sollten folgende Fragen beantwortet werden:
	\begin{itemize}
		\item Welche Services sollen wem angeboten werden?
		\item Wie unterscheiden wir uns vom Wettbewerb?
		\item Wie erzeugen wir echten Nutzen für unsere Kunden?
		\item Wie definieren wir Servicequalität?
		\item Wie finden wir den richtigen Weg zur Serviceoptimierung?
		\item Wie gestalten wir die Services wirtschaftlich?
	\end{itemize}
	\item[Service Design:] Diese Gruppe befasst sich mit der Definition und Entwicklung des Services aus der operativen Sicht. Sie setzt die Vorgaben aus der Service Strategy um. Das Service Design liefert Anleitungen für das Design und die
	Erstellung von Services und Service-Management-Prozessen. Es wird vor allem auf folgende Aufgabenbereiche eingegangen:
	\begin{itemize}
		\item Planung und Gestaltung neuer und veränderter Services
		\item Service-Management-Systeme und Tools, wie Service Portfolio und Servicekatalog
		\item Planung und Gestaltung von Technologie und Architektur
		\item Planung und Gestaltung der benötigten Prozesse
		\item Planung und Gestaltung von Messmethoden und Metriken
	\end{itemize}
	\item[Service Transition:] Die Serviceinbetriebnahme beschäftigt sich mit der Einführung eines IT-Services. Stellt eine Anleitung für den Übergang der Services in die Business-Umgebung bereit. Die wichtigsten Aufgaben und Ziele sind:
	\begin{itemize}
		\item Erkennen und Steuern der Kundenerwartungen bezüglich neuer und geänderter Services
		\item Übereinstimmung neuer oder geänderter Services mit den in den Service Requirements spezifizierten Anforderungen und Sachzwängen
		\item Integration neuer oder geänderter Services in den Business-Prozess des Kunden
		\item Serviceveränderungen werden in Bezug auf Kosten, Zeit und Qualität überwacht und gesteuert
		\item Die effektive Umsetzung der definierten Servicestrategie in den Betrieb der Services ist sichergestellt
	\end{itemize}
	\item[Service Operation:] Der Servicebetrieb beschreibt die Tätigkeiten welche notwendig sind um die vereinbarten Leistungen störungsfrei zu betreiben. Die Service Operation beschäftigt sich mit dem Tagesgeschäft (Prozesse werden täglich ausgeführt). Zur Service Operation gehören folgende Aspekte:
	\begin{itemize}
		\item Monitoring und Reporting zur optimierten Entscheidungsfindung beim Steuern von Verfügbarkeit, Nachfrage, Kapazität und aller weiteren Belange des Betriebs
		\item Sammeln und Bereitstellen von Informationen, Rückmeldungen und Ideen als Basis für den kontinuierlichen Verbesserungsprozess
		\item Sicherstellen der Verfügbarkeit und Stabilität der Services
		\item Bearbeiten und Beseitigen von Incidents und Problems
	\end{itemize}
	Der Service Betrieb bringt zahlreiche Zielkonflikte mit sich. Möchte man z.B. einen stabilen Service leidet die Flexibilität des Services darunter. Es werden folgende Zielkonflikte unterschieden:
	\begin{description}
		\item[Stabilität vs. Flexibilität:] Die IT muss sicherstellen das ein Service verfügbar ist und gleichzeitig die Änderungen des Business berücksichtigen. Je mehr Änderungen an einem Service vorgenommen werden desto mehr Serviceausfälle wird es geben. Es muss eine Balance zwischen Stabilität und Reaktionsfähigkeit gefunden werden damit der Service akzeptiert wird.
		\item[Qualität vs. Kosten:] Eine Steigerung der Qualität hat auch eine Steigerung der Kosten zur Folge. Je besser die Qualität wird desto mehr steigen die Kosten (z.B. Verfügbarkeit von 80\% auf 90\% günstiger als von 98\% auf 99\%). Durch Optimierungen z.B. von Prozessen kann die Qualität bei gleichbleibenden Kosten gesteigert werden.
		\item[Reaktiv vs. Proaktiv:] Ein reaktiver Service Provider handelt erst wenn er zum Handeln gezwungen wird. Ein proaktiver Service Provider sucht ständig Wege zur Verbesserung. Um gute Ergebnisse zu erzielen ist eine Balance zwischen reaktivem und proaktivem Verhalten zu finden. Um ein proaktive Organisation zu erhalten sind folgenden Faktoren von Bedeutung:
		\begin{itemize}
			\item Reifegrad (Erfahrung mit Bereitstellung von IT Services)
			\item Kultur (Innovation fördern)
			\item Rolle der IT (Mitglied der Geschäftsleitung)
			\item Knowledge Management (Entscheidungsgrundlage)
		\end{itemize}
	\end{description}
	\item[Continual Service Improvment:] Diese Gruppe beschäftigt sich mit der Optimierung der Service Qualität um eine bessere Wertschöpfung zu erhalten. Folgende Voraussetzungen müssen erfüllt sein:
	\begin{itemize}
		\item Die Ergebnisse des Service Level Mgmt. sind betrachtet und analysiert
		\item Benötigte Anpassungen zur Verbesserung der IT-Servicequalität sowie der Prozesseffizienz und -effektivität sind identifiziert und implementiert
		\item Die Balance zwischen wirtschaftlicher Erbringung der IT-Services und Kundenzufriedenheit ist gewährleistet und wird ständig verbessert
	\end{itemize}
\end{description}

\section{Sie wissen, was ein IT-Service ist und kennen die Bedeutung eines IT-Servicekatalog}

Ein IT-Service besteht aus einer Menge von IT-Funktionen (Hardware, Software usw.), welche einen Geschäftsprozess unterstützen. 

Früher war die IT aufgeteilt in Systeme (z.B. Anwendungen, Mainframe usw.). Aktuell bevorzugt man ein prozessorientiertes Service Management. Dabei werden gleichartige Funktionen welche in allen Systemen anfallen zu Prozessen zusammengefasst. Daraus entstehen dann Prozesse wie Change Management, Capacity Mangament usw. Dieser Wandel wurde vollzogen damit man besser auf Kundenbedürfnisse eingehen kann und wegkommt von einer technologieorientierten IT.

In einem Servicekatalog sind sämtliche IT-Services beschrieben. Er wird meistens in einen kundenorientierten Teil und in einen technologieorientierten Teil gegliedert.

\section{Support Levels}
In kleinen Betrieben werden die Levels oft vermischt.
Bei Fehlern gehen oft viele Meldungen zum selben Problem ein. Daher ist das Kategorisieren und Priorisieren so wichtig.
\subsection{1st Level Support}
Arbeitet maximal 0.5h an einem Problem. Soll 80% aller Fälle bearbeiten können (geht nur, wenn eine Knowledghebase aufgebaut wurde).
24h besetzt.

\subsection{2nd Level Support}
IT Engineers im IT Betrieb. Unterhält als Hauptjob Server, Netzwerke etc. Er arbeitet ohne Zeitlimit am Problem.

\subsection{3d Level Support}
Entwickler (Inhouse Entwickler) werden geweckt. Bei externen Dienstleistern (z.B. Oracle) wird zusammen mit den Inhouse Entwicklern nach Lösungen genutzt.