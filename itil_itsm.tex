\section{Support Levels}
In kleinen Betrieben werden die Levels oft vermischt.
Bei Fehlern gehen oft viele Meldungen zum selben Problem ein. Daher ist das Kategorisieren und Priorisieren so wichtig.
\subsection{1st Level Support}
Arbeitet maximal 0.5h an einem Problem. Soll 80% aller Fälle bearbeiten können (geht nur, wenn eine Knowledghebase aufgebaut wurde).
24h besetzt.

\subsection{2nd Level Support}
IT Engineers im IT Betrieb. Unterhält als Hauptjob Server, Netzwerke etc.
Er arbeitet ohne Zeitlimit am Problem.

Standby.

\subsection{3d Level Support}
Entwickler (Inhouse Entwickler) werden geweckt. Bei externen Dienstleistern (z.B. Oracle) wird zusammen mit den Inhouse Entwicklern nach Lösungen genutzt.

Standby.

\subsection{title}
\section{Lernkontrolle}


\subsection{Was heisst und was ist ITIL?}

Information Technology Infrastructure Library
ITIL ist ein best practice Framework für das Management von IT Applikationen, resp. IT Services. Heute ist es ein de-facto Standard. Es ist nicht organisationsspezifisch, beinhaltet ein Rollenmodell und hat den Anspruch, vollständig und umfassend zu sein. Es ist eher auf grössere Unternehmen ausgelegt und lenkt den Fokus auf den Kunden und weniger auf die konkrete Techik.

\subsection{Wieso fokussiert ITIL auf das Service Management, was ist die Grundidee dahinter?}


\subsection{Beschreiben sie die 3 Zustände}
Normal Operation
Change Operation





\subsection{Was ist ein IT-Service?}

Eine Menge von IT-Funktionen für die Unterstützung eines Business-Prozesses. 